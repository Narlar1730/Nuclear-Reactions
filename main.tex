\documentclass{article}

\usepackage[utf8]{inputenc}
\usepackage{geometry}
\usepackage[siunitx]{circuitikz}
\usepackage{pgfplots}
\usepackage{float}

\geometry{a4paper, portrait, margin=1in}

\begin{document}

\title{Nuclear Reactions}
%Howdy daniel. This is latex, and I will give you a brief run down of how to use it.
% Just writing out normally will appear as normal text. If there is a percentage in front of the writing it will be a comment, like this. NOTE HOW IT DOES NOT SHOW UP IN THE DOCUMENT ON THE RIGHT!! Please replace your UID below, it is all you need to do for this part.
\author{u5824634, u5823404}
\maketitle
\begin{center}
\section*{Abstract}
Put some shizzle wizzle in here
\end{center}
\newpage
% Below we have the first section which is the introduction. Just write whatever you want to put in there as introductions typically go.
\section{Introduction}
The fusion process is a process which allows for new elements to be created. Physicists try to control these processes every day in accelerator facilities through the bombardment of chosen projectile and target particles to create new elements. This process is very expensive and will have many variables affecting the outcomes of fusion reactions and high energy scattering events which is why understanding these processes is the key if the fusion probability is to be maximized. So, in order to understand the best probability to synthesis reaction products \footnote{reaction products are the nuclei formed in a reaction, (prelab exercise, (2017))} understanding the choice of projectile, target and beam energy is required. 

Understanding the reaction channels is important in finding the which reaction products are possible. This includes the understanding of decay. Every element that is heavier than lead demonstrates instability, which means that the particles decay exponentially. Elements that demonstrate this decaying property have a significant amount of practical uses in the real world, ranging from medical physics\footnote{There are so many different uses of radioactive materials in medicine, ranging from radioactive iodine to treat thyroid problems and barium to see how blood flows through the body} to energy generation\footnote{Energy generation is one of the most well known uses of radioactive isotopes, such as nuclear power plants.}. As such, understanding the behavior and properties of nuclear materials are critical to improving and advancing technology in the real world. 

This decay process is an exponential process of the number of fission decay products over time. The most common decay products include alpha, beta and gamma radiation. These radiations will have certain energies depending on their respective parent nuclei. This property of decay allows for the measurement of such energies which can then be corresponded to their respective parent particle. Alpha decay which is the decay of 4He have the most distinctive and measurable energies. Each decay process is related to some decay constant $\lambda$ which is the constant probability for decay per unit time which is thus related to the decay half-life which is unique for characteristic particles.
Another form of decay which can occur in a nuclear reaction, is the evaporation of protons and neutrons in some reaction channels.  This generally occurs in high energy interaction between particles when nucleons are stripped off parent particles which will affect the outcome of the reaction products.

Being able to measure what happens at bombardment will allow the measurement of the cross section of each reaction process. In other words, the relative probability for the reaction to occur in nucleon-nucleon scattering. This probability is measured in barns or area squared, and is related to the current of incident particles per unit time, the number of target particles per unit area and the rate of outgoing particles. By finding cross sections for measured half-lives, the probability of reaction to occur can be found and so by manipulating variables such as beam energy, chosen projectile and targets, this cross- section can be maximised for the desired fusion process. 

One important property of the projectile is how readily it breaks up into its constituent’s parts. For example, 7Li breaks up very easily into the alpha particle (4He) and a triton (3H), as shown in the lab manual (2017). Hence a comparison of 7Li+209Bi reaction with reactions using the other properties and to theoretical models will provide an understanding of the importance of this property

In this experiment, the reaction between 7Li +209Bi will be studied through the analysis of the alpha particles emitted by the reaction products. Through understanding this reaction, the cross sections of all the decay products will be found which will help determine the probabilities of a given fusion reaction at some high beam energy. Cross sections will be derived in terms of the variables above which can be measured.
Understanding the reaction channels is important in finding the which reaction products are possible. This includes the understanding of radioactive decay.
\section{Method}
This experiment was undertaken in many steps. Before we could undertake the experiment it was important to set up the detector correctly. TO do this we used a solid state vacuum pump which was cooled with liquid nitrogen. This placed a vacuum seal over the detector to limit uncertainty in our detector from background radiation. To start off with we needed to calibrate the detector using a known source of radiation. To do this as accurately as possible we tested three radioactive sources simultaneously of known radioactivity. In this case we used $^{241}Am$, , $^{244}Cm$ and $^{239}Pu$ to calibrate our detector. This way we get three peaks in order to develop our calibration curve to accurately measure our results later.

After this we mounted our bismuth source onto a block - taking particular care in this instance as the sample was particularly fragile. It was mounted to a copper block and prepared for the interaction with the particle accelerator beam. It was then placed into the reaction chamber and exposed to a beam of Lithium-7 ions at 45MeV for twenty minutes.

After this was finished the source was left to cool down, because it was producing high energy radiation, to the point where the Geiger Counter used was off the charts. Unfortunately this led to the destruction and decay of some of the shorter half isotopes produced, but it was necessary as it was dangerously radioactive, we needed to give it time to emit all of this energy.

After this the source was removed from the accelerator and sealed inside the detector, again reduced to a vacuum. It was important at this point to begin measuring the spectrum immediately, as the sources had already begun decaying, and some had already deteriorated. So, we collected our spectrum for five minutes in order to develop our initial spectrum. This was then done several times, with each spectrum recorded. After an hour had passed, the length that data was being collected for extended from 5 minutes to twenty minutes. This was because everything we could detect at this point had to of had a half life of greater than a few minutes, meaning it would take longer for them to decay. This means if we increase our measurement period, we can increase the likelihood of detecting these particles that take longer to decay.

Similarly, after this was repeated several times, we increased the measurement period again. This time, we increased it to a few hours repeated across a whole day in order to attempt to discover the isotopes with day long half lives. Finally, we had a final test where the detector was left running for a full six hours. This way we can try to gather data on everything that is produced in our reaction. 
\section{References}
\begin{enumerate}
\item The Synchrotron - A proposed High Energy Particle Accelerator, Edwin M. Mcmillan, University of California, Berkeley, California, September 5, 1945 \label{ref:1}
\end{enumerate}
\end{document}
